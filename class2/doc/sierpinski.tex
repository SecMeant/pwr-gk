\documentclass{article}
\usepackage[T1]{fontenc}
\usepackage[utf8]{inputenc}
\usepackage{babel}
\usepackage{listings}
\usepackage{color}
\usepackage{ragged2e}

\definecolor{dkgreen}{rgb}{0,0.6,0}
\definecolor{gray}{rgb}{0.5,0.5,0.5}
\definecolor{mauve}{rgb}{0.58,0,0.82}

\lstset{
  language=C++,
  aboveskip=3mm,
  belowskip=3mm,
  showstringspaces=false,
  columns=flexible,
  basicstyle={\small\ttfamily},
  numbers=none,
  numberstyle=\tiny\color{gray},
  keywordstyle=\color{blue},
  commentstyle=\color{dkgreen},
  stringstyle=\color{mauve},
  breaklines=true,
  breakatwhitespace=true,
  tabsize=4
}

\title{Grafika Komputerowa}
\author{Patryk Wlazłyń}

\begin{document}

\maketitle

\section{Wprowadzenie - rysowanie kwadratu}
\justify
Pierwszym zadaniem do wykonania na laboratoriach było narysowanie kwadratu używając tylko trójkątów.
Zadanie trywialne, sprowadzało się do narysowania dwóch trójkątów prostokątnych ze wspólną krawędzią w miejscu
ich przeciwprostokątnej. Żeby bezpośrednio oddać intencje programisty i nie musieć rysować dwóch trójkątów
można użyć funkcji glBegin(GL\_TRIANGLE\_STRIP). Dodatkowo została wprowadzona struktura opisująca kwadrat.
Jej definicja znajduje się poniżej.
\begin{lstlisting}
struct Square
{
  GLfloat midx, midy, rad;
  Color color;
};
\end{lstlisting}
Struktura przechowuje informację o położeniu środka kwadratu oraz jego "promieniu".
Autor jest świadomy nie poprawnego nazewnictwa zmiennej która przechowuje połowę wartości długości boku kwadratu,
jednakże została ona wybrana ze względu na dobrze kojarzącą się nazwę z przechowywaną wartością.
Poniżej zamieszczona została funkcja rysująca kwadrat.
\begin{lstlisting}
void DrawSquare(Square sq)
{
  glColor3f(sq.color.r, sq.color.g, sq.color.b);
  glBegin(GL_TRIANGLE_STRIP);
    glVertex2f(sq.midx - sq.rad, sq.midy - sq.rad);
    glVertex2f(sq.midx - sq.rad, sq.midy + sq.rad);
    glVertex2f(sq.midx + sq.rad, sq.midy - sq.rad);
    glVertex2f(sq.midx + sq.rad, sq.midy + sq.rad);
  glEnd();
}
\end{lstlisting}

\end{document}
