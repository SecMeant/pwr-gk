\documentclass[polish,polish,a4paper]{article}
\usepackage[T1]{fontenc}
\usepackage[utf8]{inputenc}
\usepackage{babel}
\usepackage{pslatex}
\usepackage{graphicx}
\usepackage{tikz}
\usepackage{pgfplots}
\usepackage{anysize}
\usepackage{pgfgantt}
\usepackage{tabularx}
\usepackage{float}
\usepackage{latexsym,amsmath}
\marginsize{2.5cm}{2.5cm}{3cm}{3cm}


\newcommand{\name}[1]{\sffamily\bfseries\scriptsize #1}

\newcommand{\frontpage}[8]{
%% #1 - nazwa kursu
%% #2 - kierunek 
%% #3 - termin 
%% #4 - temat 
%% #5 - problem
%% #6 - data

\vspace{2cm}

\begin{tabular}{|p{0.72\textwidth}|p{0.28\textwidth}|}
\hline
\multicolumn{2}{|c|}{}\\
\multicolumn{2}{|c|}{{\LARGE #1}}\\
\multicolumn{2}{|c|}{}\\
\hline
\name{Kierunek} & \name{Termin}\\
\multicolumn{1}{|c|}{\textit{#2}} & \multicolumn{1}{|c|}{\textit{#3}} \\
\hline
\name{Imię i nazwisko} & \name{Prowadzący}\\
\multicolumn{1}{|c|}{\textit{#4}} & \multicolumn{1}{|c|}{\textit{mgr inż. Szymon Datko}} \\
\hline
\end{tabular}

}

\usepackage{listings}
\usepackage{xcolor} % for setting colors

% set the default code style
\lstset{ % General setup for the package
	basicstyle=\small,
	numbers=left,
	frame=tb,
	tabsize=2,
	columns=fixed,
	showstringspaces=false,
	showtabs=false,
	keepspaces,
	commentstyle=\color{red},
	keywordstyle=\color{blue}
}

\title{Sprawozdanie SPD}
\begin{document}
% #1 - nazwa kursu #2 - kierunek  #3 - termin #4 - temat #5 - problem #6 - data
\frontpage{Grafika komputerowa i komunikacja człowiek-komputer}{Informatyka}{Poniedziałek parzysty 11:00}{Patryk Wlazłyń}
\pagestyle{empty}
\newpage

\pagebreak

\begin{table}[ht]
	\centering
	\begin{tabular}{|>{\centering\arraybackslash}p{0.72\textwidth}|>{\centering\arraybackslash}p{0.28\textwidth}|}
	\hline
	Temat ćwiczenia & Data \\
	\hline
	OpenGL - podstawy & 28.10.2019 \\
	\hline
	\end{tabular}
\end{table}

\section{Opis ćwiczenia}
Ćwiczenie polegało na poznaniu elementarnych operacji dostarczanych dzięki standardowi oraz bibliotece graficznej OpenGL. Jako bibliotekę pomocną do 
tworzenia i zarządzania oknami użyliśmy również biblioteki GLUT (GL Utility Toolkit). Podczas tego ćwiczenia poznaliśmy jak zainicjalizować bibliotekę,
jakie są różnice między używaniem jej na platformie Windows oraz Linux oraz jak generować obraz za pomocą wbudowanych prymitywów w przestrzeni 2D.
Naszym głównym zadaniem do wykonania było narysoowanie dywanu Sierpińskiego, zarówno za pomocą algorytmu rekurencyjnego oraz iteracyjnego. Dodatkowo
miał być on zbudowany z bardzo małych, różnokolorowych kwadratów z drobnymi zniekształceniami.

\subsection{Rozwiązanie rekurencyjne}
Dywan Sierpińskiego składa się z jednego kwadratu w którym wycinane są dziury w postaci mniejszych kwadratów których bok jest dokładnie 3 razy mniejszy i
rozstawione symetrycznie po każdej ze stron (lewo, prawo, góra, dół) oraz pomiędzy nimi, w rogach (górny-lewo, górne-prawo, dolne-lewo, dolne-prawo).
Łatwo zauważyć, że struktura ta powstaje w takim razie w bardzo ścisłym porządku. Ważnym spostrzeżeniem jest też, że wycinania dokonujemy na obszarach
wielkości dokładnie tej samej co środkowy (głowny) kwadrat, lecz każdy z wyciętych kwadratów jest zwyczajnie 3 razy mniejszy. To pozwala na implementacje
trywialnego algorytmu rekurencyjnego polegającego na wycięciu głównego kwadratu w środku aktualnego obszaru a następnie wycięcie mniejszych, ale zwyczajnie
mniejszch, przesuniętych wzgledem głownego o jego szerokość. Taką procedurę wystarczy powrótrzyć dla każdego nowo utworzonego (mniejszego) z kwadratów
dodając dodatkowo maksymalny poziom zagłębienia i otrzymujemy procedurę rekurencyjną generującą dywan Sierpińskiego.

\subsection{Rozwiązanie iteracyjne}
Rozwiązanie iteracyjne, najłatwiej jest rozwiązać poprzez spojrzenie na gotowy, wygenerowany dywan Śierpińskiego i zauważenie, że "dziury" konkretnych
wielkości rozstawione są w równych odstępach od siebie i występują regularnie na całej powierzchni dywanu. Ważną informacją jest to, że wieksze dziury
wydają się "przykrywać" te mniejsze. Łącząc te dwie informacje otrzymujemy dwie procedury do wykonania, aby narysować dywan. Równomierne pokrycie 
największego kwadratu dziurami oraz należy zacząć od dziur najmniejszych, tak aby te później rysowane (większe) nadpisały obraz wygenerowany przez
wcześniejsze (mniejsze).

\subsection{Dodatki}
Losowanie koloru zostało osiągniete poprzez losowanie trzech składowych koloru za pomocą funkcji rand z biblioteki standardowej języka C.
Do rysowania kwadratów została użyta funkcja rysująca dwa przystające do siebie trójkąty. Drobne zniekształcenia, wspomniane w opisie, zostały wprowadzone
poprzez losowanie ich zaraz przed narysowaniem kwadratu - funkcja rysująca kwadrat wprowadza zniekształcenia. Podobnie zostało rozwiązane rysowanie
kwadratów kolorowych.

\pagebreak
\section{Implementacja - kod źródłowy}
\lstinputlisting[caption=Dywan Sierpińskiego,language=C++]
{class1.cc}
\pagebreak

\bibliographystyle{abbrv}
\bibliography{ref}
\end{document}

